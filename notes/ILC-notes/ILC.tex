\documentclass{article}

\usepackage{amsmath}
\usepackage{amsfonts}
\usepackage{physics}

\setlength{\parindent}{0pt}

\newcommand{\isopsi}{\tilde \psi}

\title{Iterative Learning Control with Measurements}
\author{Aaron Trowbridge}
\date{}

\begin{document}

\maketitle

\section*{Setup}

Given a nominal state trajectory $\hat{x}(t)$ and control trajectory $u(t)$, we apply the controls to the experimental system and retrieve a set of measurements (abusing notation) $y(t)$ from the (possibly hidden) experimental trajectory $\bar x (t)$. Schematically we have:

$$
u(t) \longrightarrow \bar x (u(t), t) \longrightarrow g(\bar x(t), t) = \bar y(t)
$$

 
which coincides with the simplified model situation:

$$
u(t) \longrightarrow \hat x (u(t), t) \longrightarrow g(\hat x(t), t) = \hat y(t)
$$


We now have two sets of measurements: 

\begin{itemize}
  \item $\hat y(t)$ : the nominal measurement
  \item $\bar y(t)$ : the experimental measurement
\end{itemize}

\section*{Problem Formulation}

Let us write

$$
\bar x (t) = \hat x (t) + \epsilon(t)
$$


where $\epsilon(t)$ is the error in the experimental trajectory.  To correct for this error, we can find a correction term $\Delta x(t)$ s.t.

$$
g(\bar x + \Delta x) = g(\hat x + \epsilon + \Delta x ) = \hat y
$$

For example, if $g(x) = x$ is the identity function, i.e. we are trying to track the trajectory:

$$
\Delta x = - \epsilon
$$

The real problem involves finding the corresponding correction to the controls: $\Delta u(t)$. This involves setting up a quadratic optimization problem.

\newpage
\subsection*{Quadratic Correction Problem}

The goal is now to go from the measurement error $\Delta y$ to a state correction $\Delta x$ and a control correction $\Delta u$ by simultaneously solving two linear systems.  Schematically:

$$
\Delta y \xrightarrow[g]{M \cdot \Delta x = \Delta y} \Delta x \xrightarrow[f]{D \cdot \Delta z = 0} \Delta u
$$

\subsubsection*{Measurement Correction to State Correction}

With $\Delta y \equiv \bar y - \hat y$, we have

\begin{align*}
  \bar y &= g(\bar x) \\
  &= g(\hat x + \epsilon) \\ 
  &\approx g(\hat x) + \nabla g(\hat x) \cdot \epsilon \\
  &= \hat y + \nabla g(\hat x) \cdot \epsilon
\end{align*}

 
which, with writing $\nabla \hat g = \nabla g(\hat x)$ yields

$$
\Delta y \approx \nabla \hat g \cdot \epsilon
$$


and since $g: \mathbb{R}^n \to \mathbb{R}^m$ where $m \leq n$, $\nabla g$ is not necessarily invertible, but we can use the Moore-Penrose pseudoinverse here to get a guess for $\epsilon$:

$$
\boxed{
\epsilon \approx \qty(\nabla \hat g)^+ \cdot \Delta y \equiv \hat \epsilon
}
$$



To tie the experimental measurements to the model measurements we require 

\begin{align*}
\hat y &= g(\bar x + \Delta x) \\ 
&\approx \bar y + \nabla g(\bar x) \cdot \Delta x \\
\end{align*}


which yields the condition

\begin{equation}
  \boxed{
  \nabla g(\bar x) \cdot \Delta x = - \Delta y
  }
\end{equation}



where

\begin{align*}
  \nabla g(\bar x)_i^j &= \nabla g(\hat x + \epsilon)_i^j \\
  & \approx  \nabla g(\hat x + \hat \epsilon)_i^j \\
  &= \nabla g(\hat x)_i^j + \sum_k \qty(\nabla^2 g(\hat x))_i^{jk} \ \hat \epsilon_k \\
  &= \nabla \hat g_i^j + \sum_{kl} \qty(\nabla^2 \hat g)_i^{jk} \ \qty(\qty(\nabla \hat g)^+)_k^l \Delta y_l 
\end{align*}

\newpage
\subsubsection*{State Correction to Control Correction}

To propagate the state correction to the control correction, we utilize the dynamics constraint, $f(z_t, z_{t+1}) = 0$, where we define the \textit{knot point} 

$$
z_t =\mqty(x_t \\ u_t)
$$


Let's write $\vb{z}_t = \mqty(z_t \\ z_{t+1})$. Then we have

\begin{align*}
0 &= f( \hat{\vb{z}}_t + \Delta {\vb{z}}_t) \\
&\approx f(\hat{\vb{z}}_t) + \nabla f(\hat{\vb{z}}_t) \cdot \Delta {\vb{z}}_t \\
\end{align*}


which yields

\begin{equation}
  \boxed{
  \nabla f(\hat{\vb{z}}_t) \cdot \Delta {\vb{z}}_t = 0
  }
\end{equation}


\subsubsection*{Putting it all together}

We seek to find the solution to

\begin{align*}
  \underset{\Delta x_{1:T},\ \Delta u_{1:T}}{\text{minimize}} & \quad \sum_t \Delta x^\top_t Q \Delta x_t + \Delta u^\top_t R \Delta u_t \\
  \text{subject to} \ & \quad \nabla g(\bar x_\tau) \cdot \Delta x_\tau = - \Delta y_\tau \quad \forall \tau \\
  & \quad \nabla f(\hat{\vb{z}}_t) \cdot \Delta {\vb{z}}_t = 0 \quad \forall t
\end{align*}


where the $\tau$s are the measurement times.

\hfill


Building the KKT matrix from this problem, we can solve the system and extract $\Delta u(t)$ and repeat the procedure until convergence.

\newpage
\subsubsection*{KKT Matrix (for just single quantum state and controls)}

Below we use:

\begin{itemize}
  \item $n = \dim z_t = \dim x_t + \dim u_t $ 
  \item $d = \dim x_t = \dim f(z_t, z_{t+1}) $ 
  \item $c = \dim u_t$  
  \item $m = \dim y_t$ 
  \item $M = \#\text{ of measurements}$
\end{itemize}



For a trajectory $Z = \text{vec}(z_{1:T})$, we need to construct the matrix

$$
\mqty(
  H & A^\top \\
  A & 0
)
$$


where $H$ is the Hessian of the cost function:

$$
H = \bigoplus_{t=1}^T \qty(Q \oplus R) = I^{T \times T} \otimes \qty(Q \oplus R)
$$


and $A$ is the constraint Jacobian:

$$
A = \mqty(
  \nabla F \\
  \nabla G
)
$$

with 

$$
\nabla F = \mqty(
  \nabla f(\hat{\vb{z}}_1) \\
  & \ddots \\
  & & \nabla f(\hat{\vb{z}}_{T-1})
) \in \mathbb{R}^{d (T - 1) \times n T}
$$

and

$$
\nabla G = \mqty(
  \ddots \\
  & \nabla g(\bar x_{\tau}) \ \vb{0}^{m \times (a + c)} \\
  & & \ddots
) \in \mathbb{R}^{m M \times n T}
$$

where $\tau = t_1, \dots, t_M$ are the measurement times.

\hfill

For the constraints we then have

$$
\nabla F \cdot \Delta Z = 0
\quad \text{and} \quad
\nabla G \cdot \Delta Z = - \Delta Y
$$

where again

$$
\Delta Y = \bar Y - \hat Y
$$
 
\end{document}